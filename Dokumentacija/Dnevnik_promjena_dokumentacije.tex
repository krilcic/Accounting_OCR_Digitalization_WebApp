\chapter{Dnevnik promjena dokumentacije}
		
		\textbf{\textit{Kontinuirano osvježavanje}}\\
				
		
		\begin{longtblr}[
				label=none
			]{
				width = \textwidth, 
				colspec={|X[2]|X[13]|X[3]|X[3]|}, 
				rowhead = 1
			}
			\hline
			\textbf{Rev.}	& \textbf{Opis promjene/dodatka} & \textbf{Autori} & \textbf{Datum}\\[3pt] \hline
			0.1 & Napravljen predložak dokumentacije.	& Marko Šelendić & 24.10.2023. 		\\[3pt] \hline 
			0.2 & Dodani funkcionalni zahtjevi i obrasci uporabe. & Marko Šelendić & 25.10.2023. 	\\[3pt] \hline 
			0.3 & Dodan opis projekta. & Tomislav Čupić & 30.10.2023. \\[3pt] \hline
			0.4 & Dodani sekvencijski dijagrami & Nika Miličević & 4.11.2023. \\[3pt] \hline 
			0.5 & Dodani \textit{Use Case} dijagrami & Nika Miličević & 4.11.2023. \\[3pt] \hline 
			0.6 & Dodani prvi sastanci & Marko Šelendić & 8.11.2023. \\[3pt] \hline 
			0.6.1 & Dodani ostali sastanci \newline & Tomislav Čupić & 9.11.2023. \\[3pt] \hline
			0.7 & Dodani opis baze & Filip Krilčić & 13.11.2023. \\[3pt] \hline
			0.8 & Dodani opis arhitekture & Zvonimir Pipić & 14.11.2023. \\[3pt] \hline
			\textbf{1.0} & Osnovni model aplikacije pušten u pogon & Svi & 17.11.2023. \\[3pt] \hline 
			1.1 & Dodane korištene tehnologije & Filip Krilčić & 4.1.2024. \\[3pt] \hline	
			1.2 & Dodani dijagrami aktivnosti i dijagrami komponenti & Nika Miličević & 9.1.2024. \\[3pt] \hline	
			1.3 & Dodano ispitivanje komponenti & Filip Krilčić & 10.1.2024. \\[3pt] \hline	
			1.3.1 & Popravljeni dijagrami & Nika Miličević & 13.1.2024. \\[3pt] \hline	
			1.3.2 & Dodan sastanak & Tomislav Čupić & 13.1.2024. \\[3pt] \hline
			1.4 & Dodane upute za puštanje u pogon & Nika Miličević & 18.1.2024. \\[3pt] \hline
			& & & \\[3pt] \hline
		\end{longtblr}
	
	
		\textit{Moraju postojati glavne revizije dokumenata 1.0 i 2.0 na kraju prvog i drugog ciklusa. Između tih revizija mogu postojati manje revizije već prema tome kako se dokument bude nadopunjavao. Očekuje se da nakon svake značajnije promjene (dodatka, izmjene, uklanjanja dijelova teksta i popratnih grafičkih sadržaja) dokumenta se to zabilježi kao revizija. Npr., revizije unutar prvog ciklusa će imati oznake 0.1, 0.2, …, 0.9, 0.10, 0.11.. sve do konačne revizije prvog ciklusa 1.0. U drugom ciklusu se nastavlja s revizijama 1.1, 1.2, itd.}