\chapter{Zaključak i budući rad}

		Zadatak naše grupe bio je izrada web aplikacije za ubrzavanje procesa digitalizacije računovodstvenih tvrtki.
		Naša aplikacija ima jednostavno korisničko sučelje koje omogućuje korisniku da brzo i jednostavno skenira i upravlja dokumentima.
		Izrada aplikacije trajala je otprilike 14 tjedana i u tom vremenu smo uspjeli ostvariti sve funkcionalnosti koje smo planirali.
		Prije same izrade aplikacije, cijeli tim se sastao kako bi se dogovorili koje alate i tehnologije ćemo koristiti pri izradi aplikacije 
		te kako bi se rasporedili u podtimove i odredili uloge svakog člana tima. Osim toga napravili smo detaljnu analizu zahtjeva 
		kako bi što bolje razumjeli što je potrebno napraviti i kako bi mogli projekt podijeliti na manje dijelove koji su se mogli razvijati paralelno
		i kako bi se mogli rasporediti zadaci svakom članu tima. Pri izradi aplikacije susreli smo se s nekoliko tehničkih izazova, ali smo ih uspješno riješili.
		Budući da je većini tima ovo prvi projekt u ovom obliku, nitko nije bio bolje upoznat s tehnologijama koje smo koristili tako da je prvi izazov bio upoznavanje s tehnologijama.
		Taj izazov smo riješili tako da smo našli nekoliko pouzdanih resursa koji su nam pomogli da se upoznamo s tehnologijama te smo jednostavno krenuli raditi i učili smo što nam 
		je bilo potrebno. S obzirom da smo prije početka izrade aplikacije napravili detaljnu analizu zahtjeva, sama izrada aplikacije je bila uvelike olakšana jer smo imali jasnu sliku što je potrebno napraviti.
		Zbog toga smo od početka izrade aplikacije bili vrlo dobro organizirani i nismo imali problema s koordinacijom između članova tima te nam je to uvelike olakšalo izvođenje traženih zahtjeva.
		Drugi izazov je bio povezivanje frontenda i backenda, odnosno slanje podataka između njih. Taj izazov smo riješili tako da smo se raspitali i pronašli smo nekoliko pouzdanih resursa koji su nam pomogli da se upoznamo s načinom na koji se to radi.
		Osim toga nije bilo većih izazova pri izradi aplikacije, sve ostalo su bile manje poteškoće na koje smo naišli tokom ostvarivanja pojedinih zahtjeva, a sve takve poteškoće je rješavao član tima
		koji je bio zadužen za taj dio aplikacije dodatnim istraživanjem i učenjem te su se po potrebi konzultirali s ostalim članovima tima.
		S obzirom da je ovo prvi projekt u ovom obliku za većinu članova tima, svi smo stekli nova znanja o tehnologijama koje smo koristili te 
		iskustvo rada u timu s kojim smo razvili vještine komunikacije i organizacije koje će nam sigurno koristiti u budućnosti.
		
		\eject 